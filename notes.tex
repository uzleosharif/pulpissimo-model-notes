



\documentclass{article}


\usepackage{enumitem, amssymb}
\newlist{todolist}{itemize}{2}
\setlist[todolist]{label=$\square$}

\usepackage{pifont}
\newcommand{\cmark}{\ding{51}}%
\newcommand{\xmark}{\ding{55}}%
\newcommand{\done}{\rlap{$\square$}{\raisebox{2pt}{\large\hspace{1pt}\cmark}}%
\hspace{-2.5pt}}
\newcommand{\wontfix}{\rlap{$\square$}{\large\hspace{1pt}\xmark}}

\usepackage[hidelinks]{hyperref}

\title{Modeling Peripherals of Pulpissimo Chip}
\author{uzleo}
\date{}

\begin{document}

\maketitle


\section{Goal}
The goal of this work is to model the peripherals, especially the $\mu$DMA based, of the Pulpissimo Chip.
The models are developed in SystemC at a fairly high abstraction level. Key thing is to ensure that this
chip model could be used for quick SW development by other teams. Currently, we see following work items:

\begin{todolist}
 % \item[\done], \item[wontfix] ...
 \item[\done] setup \texttt{TGC-VP} simulation on Ubuntu20 (EDA machine)
 \item[\done] create Pulpissimo project in \texttt{VP-Vibes} space and push modified \texttt{TGC-VP} in this space.
 \item remove TGC-VP specific components
 \item[\done] adapt bare-metal SW compilation flow as per Pulpissimo's RISC-V architecture
 \item[\done] insert bare-bone peripheral on the address of $\mu$DMA and make sure this can be accessed on a
 dummy register via SW
 \item SPI-M model using Minres API
 \item I2S model using Minres API
 \item[\done] pull request for the peripherals in \texttt{VPV-Peripherals}
 \item[\done] ask Axel for demo binary. What about runtime environment: is it bare-metal or based on PULP SDK?
\end{todolist}

\section{Progress}
\begin{itemize}
 \item forked \texttt{TGC-VP} from Minres \texttt{TGC-VP:develop} to create a new \texttt{Pulpissimo-VP} project in
       \texttt{VP-Vibes} space
 \item made small changes to the \texttt{README} file to be compatible with new project space
 \item inserted a dummy-peripheral and accessed it in SW
 \item bare-bone UDMA peripheral (with in \texttt{VPVPer}) integration in TGC-VP now working.
 \item github project organization: \texttt{Pulpissimo-VP} project in \texttt{VP-Vibes} space but this uses
       submodule \texttt{VPVPer} in \texttt{uzleosharif} space as I only have write access to my repos
\end{itemize}

\section{Notes on SPI-M model}
\subsection{Tx side}

\subsection{Rx side}
Involves three registers: \texttt{SPIM\_RX\_SADDR, SPIM\_RX\_SIZE, SPIM\_RX\_CFG}. These are used to configure the RX
channel of uDMA (between uDMA and SPIM IP) to receive the data from SPIM IP to the system memory.

\noindent \rule[0.5ex]{\linewidth}{1pt}

\noindent \textbf{How data moves from external device to system memory via SPIM interface?}

The SPIM is always the master of SPI communication hence the external device can't initiate the communication. The
SPIM IP would be the first to request data from external device (the slave in SPI protocol).

So my guess is the
CPU first programs the uDMA that it needs to receive data from outside via SPI commands. These commands are then
processed by uDMA which in turn programs the SPIM IP to initiate the Rx request from external device. After
successful data transfer, the SPIM IP buffers all the received data. \textit{Then uDMA (or SPIM IP)
 signals (creates event) to the CPU}.
The CPU can then configure the RX channel (\texttt{SPIM\_RX\_SADDR, SPIM\_RX\_SIZE, SPIM\_RX\_CFG})
to get the data from SPIM IP to uDMA (via uDMA-SPIM-RX channel) to the system memory.

\noindent \textbf{TODO:} read SPIM (bare-metal/PULP-SDK) example code to make sure above flow is correct.

\noindent \rule[0.5ex]{\linewidth}{1pt}

The \texttt{pulp-rt} driver function \texttt{rt\_spim\_receive()} states that "Due to hardware constraints, the address
of the buffer must be aligned on 4 bytes and the size must be a multiple of 4"

\subsection{Commands}

\subsection{SW}
To better understand SPIM functionality we also look at various SW (applications) that are available on the internet.
So far I found following useful:
\begin{itemize}
 \item \texttt{pulp-rt-examples} in \texttt{pulp-platform} esp. the async example. This uses \texttt{pulp-rt}.
       % \item \texttt{pulp-runtime} in \texttt{pulp-platform} but only drivers.
 \item  \texttt{pmsis\_tests} in \texttt{GreenWaves-Technologies}
 \item \texttt{pulp-freertos} in \texttt{pulp-platform} provides spi stuff in tests
\end{itemize}


\section{Notes on I2S model}

\section{Issues}
\begin{itemize}
 \item Conan has to be upgraded to $>$= 1.51.3 for compiling \texttt{TGC-VP / Pulpissimo-VP}.
 \item \texttt{dbt-rise-core} latest commits are not compilable (gcc 9.4). Stick to old commit \textbf{0x44acf8a}.
       This is also now noted down in READ file as well.
 \item some bugs in generated regs map for UDMA peripherals (\texttt{VPVPer} project):
       (i) some wrong base-addresses (ii) some wrong offset calculations
\end{itemize}

\section{SW development for Pulpissimo-VP}
\subsection{Bare-metal}
The Pulpissimo VP simulates RISCY core (\texttt{RV32IMC/F}). For SW development, we use
TGC-VP compilation flow. This is a \textit{Makefile} based system that requires \texttt{riscv32-unknown-elf}
command to be available on terminal.

\subsection{PULP SDK}

\newpage
\section{Useful Repositories}
\begin{itemize}
 \item \texttt{VP-Vibes} Project
 \item \texttt{SystemC-Components} is a SystemC library used to quickly build VPs
 \item \texttt{VPV-Peripherals} is a dedicated repo to hold peripherals that are built using VP-Vibes concept
 \item \texttt{TGC-VP} Scale4Ege ecosystem VP
 \item \texttt{pulp-runtime} in \texttt{pulp-platform} space which is a simple runtime for Pulp architecture
 \item \texttt{pulp-runtime-examples} in \texttt{pulp-platform} provides some examples that on how to use
       \texttt{pulp-runtime} project
 \item \texttt{pulp-freertos} in \texttt{pulp-platform} space which provides FreeRTOS support and drivers for
       development of real-time applications on PULP based systems.

        [Paul (Scale4Edge)] it is up-to-date, has most features, and is rather easy to use. Unfortunately, it is
       missing I2S drivers.
 \item \texttt{custom-pulp-sdk} in EKUT project space

       gitlab: \url{https://atreus.informatik.uni-tuebingen.de}

       [Paul (Scale4Edge)] our own port ofthe PulpSDK. It is based on an old version, but it is fully functional
       and contain all drivers.

       It seems to me that Axel Sauer (Bosch) is also developing his binaries using this runtime.
 \item \texttt{pulp-sdk} in \texttt{pulp-platform} that serves as a pre-cursor to EKUT SDK. However, latest
       Pulpissimo is incompatible with this SDK and recommendation is to use FreeRTOS port instead.
 \item \texttt{pulp-rt-examples} in \texttt{pulp-platform} gives some examples on how to use \texttt{pulp-sdk}
       % \item \texttt{pmsis\_tests} in \texttt{GreenWaves-Technologies} also gives some examples on how to use
       %       \texttt{pulp-sdk} and is more up-to-date and recommended by \texttt{pulp-sdk} project.
\end{itemize}

\end{document}

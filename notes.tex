



\documentclass{article}


\usepackage{enumitem, amssymb}
\newlist{todolist}{itemize}{2}
\setlist[todolist]{label=$\square$}

\usepackage{pifont}
\newcommand{\cmark}{\ding{51}}%
\newcommand{\xmark}{\ding{55}}%
\newcommand{\done}{\rlap{$\square$}{\raisebox{2pt}{\large\hspace{1pt}\cmark}}%
\hspace{-2.5pt}}
\newcommand{\wontfix}{\rlap{$\square$}{\large\hspace{1pt}\xmark}}

\usepackage[hidelinks]{hyperref}

\usepackage{listings}

\title{Modeling Peripherals of Pulpissimo Chip}
\author{uzleo}
\date{}

\begin{document}

\maketitle


\section{Goal}
The goal of this work is to model the peripherals, especially the $\mu$DMA based, of the Pulpissimo Chip.
The models are developed in SystemC at a fairly high abstraction level. Key thing is to ensure that this
chip model could be used for quick SW development by other teams. Currently, we see following work items:

\begin{todolist}
 % \item[\done], \item[wontfix] ...
 \item[\done] setup \texttt{TGC-VP} simulation on Ubuntu20 (EDA machine)
 \item[\done] create Pulpissimo project in \texttt{VP-Vibes} space and push modified \texttt{TGC-VP} in this space.
 \item[\done] remove TGC-VP specific components --- Adapt memory, interrupt-controller to Pulpissimo SoC
 \item[\done] adapt bare-metal SW compilation flow as per Pulpissimo's RISC-V architecture
 \item[\done] insert bare-bone peripheral on the address of $\mu$DMA and make sure this can be accessed on a
 dummy register via SW
 \item[\done] Pulpissimo SoC using Minres API
 \item SPIM model using Minres API
 \item I2S model using Minres API
 \item[\done] pull request for the peripherals in \texttt{VPV-Peripherals}
 \item[\done] ask Axel for demo binary. What about runtime environment: is it bare-metal or based on PULP SDK?
 \item how to notify from peripheral to CPU that an event occurred in Minres VPs? Maybe ask Eyck
 \item change DBT core to RISCY core that is used in Pulpissimo SoC. Possibly with Eyck help
 \item[\done] Compile \texttt{custom-pulp-sdk}
 \item make sure that empty main() proram is running on Pulpissimo VP
 \item[\done] SoC-Control needed for bootstrapping
 \item[\done] how to set reset values on Minres peripherals?
 \item[\done] it seems that `custom-pulp-sdk` is out-dated with latest Pulpissimo RTL and the peripheral mmap is not
 matching with latest datasheet spec. Clear this up with Paul what is the peripheral mmap they are using in their RTL
 so I can be compatible to that model.
 \item RDL flow for pulpissimo peripherals .. dont modify generated files rather modify the RDL templates
 \item[\done] SDK bug as we always get the interrupt vector table base to be address \texttt{0x0} instead of possibly
 address in L2 (\texttt{0x1c00\_8000}). Notify Paul
\end{todolist}

\section{Progress}
\begin{itemize}
 \item forked \texttt{TGC-VP} from Minres \texttt{TGC-VP:develop} to create a new \texttt{Pulpissimo-VP} project in
       \texttt{VP-Vibes} space
 \item made small changes to the \texttt{README} file to be compatible with new project space
 \item inserted a dummy-peripheral and accessed it in SW
 \item bare-bone UDMA peripheral (with in \texttt{VPVPer}) integration in TGC-VP now working.
 \item github project organization: \texttt{Pulpissimo-VP} project in \texttt{VP-Vibes} space but this uses
       submodule \texttt{VPVPer} in \texttt{uzleosharif} space as I only have write access to my repos
 \item made it till \texttt{\_\_rt\_time\_init} in startup code
 \item now able to enter \texttt{main} albeit with incomplete peripherals and also some workarounds in runtime (due to no RI5CY core). So far
       only some periphreals' regs are modeled (that too incompletely) just to hit main function
\end{itemize}

\section{Notes on Pulpissimo SoC model}
\begin{itemize}
 \item for timer model, have a look at \texttt{sifive::clint} peripheral in \texttt{TGC-VP}
\end{itemize}

\section{Notes on SPIM model}
\subsection{Tx side}

\subsection{Rx side}
Involves three registers: \texttt{SPIM\_RX\_SADDR, SPIM\_RX\_SIZE, SPIM\_RX\_CFG}. These are used to configure the RX
channel of uDMA (between uDMA and SPIM IP) to receive the data from SPIM IP to the system memory.

\noindent \rule[0.5ex]{\linewidth}{1pt}

\noindent \textbf{How data moves from external device to system memory via SPIM interface?}

The SPIM is always the master of SPI communication hence the external device can't initiate the communication. The
SPIM IP would be the first to request data from external device (the slave in SPI protocol).

So my guess is the
CPU first programs the uDMA that it needs to receive data from outside via SPI commands. These commands are then
processed by uDMA which in turn programs the SPIM IP to initiate the Rx request from external device. After
successful data transfer, the SPIM IP buffers all the received data. \textit{Then uDMA (or SPIM IP)
 signals (creates event) to the CPU}.
The CPU can then configure the RX channel (\texttt{SPIM\_RX\_SADDR, SPIM\_RX\_SIZE, SPIM\_RX\_CFG})
to get the data from SPIM IP to uDMA (via uDMA-SPIM-RX channel) to the system memory.
\\\\
Looking at older PulpSDK version, we see that CPU initiates the transfer using
%
\begin{lstlisting}[language=C]
rt_spim_receive(
  spim, rx_buffer, BUFFER_SIZE * 8, RT_SPIM_CS_AUTO, NULL
);

/** \brief Enqueue a read copy to the SPI
 * (from SPI to Chip device).
 *
 * This function can be used to receive data from the
 * SPI device. The copy will make an asynchronous transfer
 * between the SPI and one of the chip memory. An event
 * can be specified in order to be notified when the
 * transfer is finished. This is using classic SPI
 * transfer with MOSI and MISO lines.
 * Due to hardware constraints, the address of the buffer
 * must be aligned on 4 bytes and the size must be a
 * multiple of 4.
 *
 * handle: The handle of the SPI device which was returned
 * when the device was opened.
 * data: The address in the chip where the received data
 * must be written.
 * len: The size in bits of the copy.
 * cs_mode: The mode for managing the chip select.
 * event: The event used to notify the end of transfer.
 * See the documentation of rt_event_t for more details.
 */
// static inline void rt_spim_receive(
//  rt_spim_t *handle, void *data, size_t len,
//  rt_spim_cs_e cs_mode, rt_event_t *event
//  );
\end{lstlisting}
%
which internally creates the uDMA commands to configure the SPIM-CMD and SPI-RX channels as following
%
\begin{lstlisting}[language=C]
int cfg = UDMA_CHANNEL_CFG_SIZE_8 | UDMA_CHANNEL_CFG_EN;
plp_udma_enqueue(channel_base, (int)data, buffer_size, cfg);
plp_udma_enqueue(cmd_base, (int)cmd, 4 * 4, cfg);

/** Enqueue a new transfer to a UDMA channel.
 *
  channelOffset: Offset of the channel
  l2Addr: Address in L2 memory where the data must be
  accessed (either stored or loaded)
  size: Size in bytes of the transfer
  cfg: Configuration of the transfer. Can be
  UDMA_CHANNEL_CFG_SIZE_8, UDMA_CHANNEL_CFG_SIZE_16 or
  UDMA_CHANNEL_CFG_SIZE_32
  */
// static inline void plp_udma_enqueue(
//   unsigned channelOffset, unsigned int l2Addr,
//   unsigned int size, unsigned int cfg
// ) {
//  pulp_write32(channelBase + UDMA_CHANNEL_SADDR_OFFSET,
//    l2Addr);
//  pulp_write32(channelBase + UDMA_CHANNEL_SIZE_OFFSET,
//    size);
//  pulp_write32(channelBase + UDMA_CHANNEL_CFG_OFFSET,
//    cfg | UDMA_CHANNEL_CFG_EN);
// }

\end{lstlisting}
%
Looking at these code snippets, it seems the the following happens for data transfer between external SPI device
to Pulpissimo system memory (L2) via SPI protocol
\begin{itemize}
 \item CPU configures the Rx channel of SPIM peripheral in uDMA and sends the Rx-buffer pointer
 \item CPU configures the CMD channel of SPIM peripheral in uDMA and sends some commands to prepare for Rx operation
 \item uDMA initiates the Rx transfer with external SPI device and autonomously fills the data in Rx-buffer (L2 mem)
 \item uDMA then triggers the event to CPU notifying it about the end of requested transfer
\end{itemize}
In this driver, following commnads were sent to uDMA from CPU via the command buffer (unsigned int cmd[4];)
\begin{enumerate}
 \item \texttt{SPI\_CMD\_CFG (0x0)} : clock for SPI communication
 \item \texttt{SPI\_CMD\_SOT (0x1)} : set chip select (slave)
 \item \texttt{SPI\_CMD\_RX\_DATA (0x7)} : how much data to receive, what is word size, how many words per transfer,
       lsb first?, qspi? etc.
 \item \texttt{SPI\_CMD\_EOT (0x9)} : clear chip select (slave)
\end{enumerate}
%
\noindent \rule[0.5ex]{\linewidth}{1pt}

The \texttt{pulp-rt} driver function \texttt{rt\_spim\_receive()} states that "Due to hardware constraints, the address
of the buffer must be aligned on 4 bytes and the size must be a multiple of 4"

\subsection{Commands}

\subsection{SW}
To better understand SPIM functionality we also look at various SW (applications) that are available on the internet.
So far I found following useful:
\begin{itemize}
 \item \texttt{pulp-rt-examples} in \texttt{pulp-platform} esp. the async example. This uses \texttt{pulp-rt}.
       % \item \texttt{pulp-runtime} in \texttt{pulp-platform} but only drivers.
 \item  \texttt{pmsis\_tests} in \texttt{GreenWaves-Technologies}
 \item \texttt{pulp-freertos} in \texttt{pulp-platform} provides spi stuff in tests
\end{itemize}


\section{Notes on I2S model}

\section{Issues}
\begin{itemize}
 \item Conan has to be upgraded to $>$= 1.51.3 for compiling \texttt{TGC-VP / Pulpissimo-VP}.
 \item \texttt{dbt-rise-core} latest commits are not compilable (gcc 9.4). Stick to old commit \textbf{0x44acf8a}.
       This is also now noted down in READ file as well.
 \item some bugs in generated regs map for UDMA peripherals (\texttt{VPVPer} project):
       (i) some wrong base-addresses (ii) some wrong offset calculations
 \item The pulpssimo mmap \texttt{0x1A10\_9000} seems to be mirror mapped to \texttt{0x1A10\_9800} as well
\end{itemize}

\section{SW development for Pulpissimo-VP}
\subsection{Bare-metal}
The Pulpissimo VP simulates RISCY core (\texttt{RV32IMC/F}). For SW development, we use
TGC-VP compilation flow. This is a \textit{Makefile} based system that requires \texttt{riscv32-unknown-elf}
command to be available on terminal.

\subsection{PULP SDK}

\newpage
\section{Useful Repositories}
\begin{itemize}
 \item \texttt{VP-Vibes} Project
 \item \texttt{SystemC-Components} is a SystemC library used to quickly build VPs
 \item \texttt{VPV-Peripherals} is a dedicated repo to hold peripherals that are built using VP-Vibes concept
 \item \texttt{TGC-VP} Scale4Ege ecosystem VP
 \item \texttt{pulp-runtime} in \texttt{pulp-platform} space which is a simple runtime for Pulp architecture
 \item \texttt{pulp-runtime-examples} in \texttt{pulp-platform} provides some examples that on how to use
       \texttt{pulp-runtime} project
 \item \texttt{pulp-freertos} in \texttt{pulp-platform} space which provides FreeRTOS support and drivers for
       development of real-time applications on PULP based systems.

        [Paul (Scale4Edge)] it is up-to-date, has most features, and is rather easy to use. Unfortunately, it is
       missing I2S drivers.
 \item \texttt{custom-pulp-sdk} in EKUT project space

       gitlab: \url{https://atreus.informatik.uni-tuebingen.de}

       [Paul (Scale4Edge)] our own port ofthe PulpSDK. It is based on an old version, but it is fully functional
       and contain all drivers.

       It seems to me that Axel Sauer (Bosch) is also developing his binaries using this runtime.
 \item \texttt{pulp-sdk} in \texttt{pulp-platform} that serves as a pre-cursor to EKUT SDK. However, latest
       Pulpissimo is incompatible with this SDK and recommendation is to use FreeRTOS port instead.
 \item \texttt{pulp-rt-examples} in \texttt{pulp-platform} gives some examples on how to use \texttt{pulp-sdk}
       % \item \texttt{pmsis\_tests} in \texttt{GreenWaves-Technologies} also gives some examples on how to use
       %       \texttt{pulp-sdk} and is more up-to-date and recommended by \texttt{pulp-sdk} project.
\end{itemize}

\end{document}





\documentclass{article}


\usepackage{enumitem, amssymb}
\newlist{todolist}{itemize}{2}
\setlist[todolist]{label=$\square$}

\usepackage{pifont}
\newcommand{\cmark}{\ding{51}}%
\newcommand{\xmark}{\ding{55}}%
\newcommand{\done}{\rlap{$\square$}{\raisebox{2pt}{\large\hspace{1pt}\cmark}}%
\hspace{-2.5pt}}
\newcommand{\wontfix}{\rlap{$\square$}{\large\hspace{1pt}\xmark}}


\title{Modeling Peripherals of Pulpissimo Chip}

\begin{document}

\maketitle


\section{Goal}
The goal of this work is to model the peripherals, especially the $\mu$DMA based, of the Pulpissimo Chip.
The models are developed in SystemC at a fairly high abstraction level. Key thing is to ensure that this
chip model could be used for quick SW development by other teams. Currently, we see following work items:

\begin{todolist}
 % \item[\done], \item[wontfix] ...
 \item[\done] setup \texttt{TGC-VP} simulation on Ubuntu20 (EDA machine)
 \item[\done] create Pulpissimo project in \texttt{VP-Vibes} space and push modified \texttt{TGC-VP} in this space.
 \item remove TGC-VP specific components
 \item[\done] adapt bare-metal SW compilation flow as per Pulpissimo's RISC-V architecture
 \item[\done] insert bare-bone peripheral on the address of $\mu$DMA and make sure this can be accessed on a
 dummy register via SW
 \item SPI-M model using Minres API
 \item I2S model using Minres API
 \item[\done] pull request for the peripherals in \texttt{VPV-Peripherals}
\end{todolist}

\section{Progress}
\begin{itemize}
 \item forked \texttt{TGC-VP} from Minres \texttt{TGC-VP:develop} to create a new \texttt{Pulpissimo-VP} project in
       \texttt{VP-Vibes} space
 \item made small changes to the \texttt{README} file to be compatible with new project space
 \item inserted a dummy-peripheral and accessed it in SW
 \item bare-bone UDMA peripheral (with in \texttt{VPVPer}) integration in TGC-VP now working.
 \item github project organization: \texttt{Pulpissimo-VP} project in \texttt{VP-Vibes} space but this uses
       submodule \texttt{VPVPer} in \texttt{uzleosharif} space as I only have write access to my repos
\end{itemize}

\section{Notes on SPI-M model}

\section{Notes on I2S model}

\section{Issues}
\begin{itemize}
 \item Conan has to be upgraded to $>$= 1.51.3 for compiling \texttt{TGC-VP / Pulpissimo-VP}.
 \item \texttt{dbt-rise-core} latest commits are not compilable (gcc 9.4). Stick to old commit \textbf{0x44acf8a}.
       This is also now noted down in READ file as well.
 \item some bugs in generated regs map for UDMA peripherals (\texttt{VPVPer} project):
       (i) some wrong base-addresses (ii) some wrong offset calculations
\end{itemize}

\section{SW development for Pulpissimo-VP}
\subsection{Bare-metal}
The Pulpissimo VP simulates RISCY core (\texttt{RV32IMC/F}). For SW development, we use
TGC-VP compilation flow. This is a \textit{Makefile} based system that requires \texttt{riscv32-unknown-elf}
command to be available on terminal.

\subsection{PULP SDK}

\newpage
\section{Useful Repositories}
\begin{itemize}
 \item \texttt{VP-Vibes} Project
 \item \texttt{SystemC-Components} is a SystemC library used to quickly build VPs
 \item \texttt{VPV-Peripherals} is a dedicated repo to hold peripherals that are built using VP-Vibes concept
 \item \texttt{TGC-VP} Scale4Ege ecosystem VP
 \item EKUT project repo that has a test program for I2S peripheral that runs on Pulpissimo Chip
\end{itemize}

\end{document}
